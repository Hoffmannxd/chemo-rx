\documentclass[a4paper,12pt,oneside,openany]{report}	
\usepackage{layout}
\setlength{\textwidth}{15.0 cm}
\setlength{\textheight}{25.0 cm}


\usepackage[brazilian]{babel}
\usepackage{pagina}	% pagina-padrao
\usepackage{indentfirst}		% for indent

\usepackage[latin1]{inputenc}
\usepackage[T1]{fontenc}

\usepackage{hyphenat}
\hyphenation{mate-mática recu-perar}

\usepackage{graphics,epsfig}
\usepackage{graphics}
\graphicspath{{./figuras/}}
\usepackage{pstricks,pst-node,pst-tree}
\usepackage{alltt}
%\usepackage{makeidx}
%\makeindex
\usepackage[figuresright]{rotating} % for saydways tables and figures
\usepackage{enumerate}			% for configuration of enumerate environment
\usepackage{amsmath}
\usepackage{amssymb}
\usepackage{portland,multirow}

\setcounter{secnumdepth}{3}	% numeracao ate subsubsecao
\setcounter{tocdepth}{2}	% indice ate subsubsecao

\usepackage{longtable}


\begin{document}

\begin{center}
\textbf{UNIVERSIDADE FEDERAL DO RIO DE JANEIRO}
\vspace{-0.2cm}

\textbf{ESCOLA POLIT\'ECNICA}
\vspace{-0.2cm}

\textbf{DEPARTAMENTO DE ENGENHARIA ELETR\^ONICA E DE COMPUTA\c{C}ÃO}
\vspace{0.8cm}

\underline{\textbf{PROPOSTA DE PROJETO DE GRADUA\c{C}ÃO}}

Aluno: Matheus Hoffmann Fernandes Santos
\vspace{-0.2cm}

hoffmann@poli.com

Orientador: Alexandre Visintainer Pino
\end{center}

\textbf{1. T\'ITULO}

An\'alise de modelos estat\'isticos preditivos para rea\c{c}\~oes adversas ao tratamento quimioter\'apico em portadoras do c\^ancer de mama.

\vspace{0.4cm}
\textbf{2. \^ENFASE}

Computa\c{c}\~ao

\vspace{0.4cm}
\textbf{3. TEMA}

O tema do trabalho \'e a an\'alise de diversos modelos para a predi\c{c}\~ao de efeitos colaterais devido ao tratamento do c\^ancer de mama, tendo como base casos anteriores em que foi poss\'ivel coletar informa\c{c}\~oes pessoais e gen\'eticas de portadoras bem como suas rea\c{c}\~oes ao tratamento, sendo elas principalmente: astenia; mialgia; artralgia; mucosite; dor abdominal e diarreia. 

\vspace{0.4cm}
\textbf{4. DELIMITA\c{C}ÃO}

O objeto de estudo do projeto s\~ao os modelos estat\'isticos que podem classificar as poss\'iveis rea\c{c}\~oes de uma dada portadora do c\^ancer em tratamento com informa\c{c}\~oes conhecidas a priori nos mesmos termos dos casos anteriores previamente usados para ajustar esses mesmos modelos.

\vspace{0.4cm}
\textbf{5. JUSTIFICATIVA}

O c\ˆancer de mama \'e o segundo tipo de c\ˆancer mais frequente no mundo e o primeiro entre as mulheres, respondendo por 22\% dos novos casos a cada ano [1].  A evolu\c{c}\~ao no tratamento para o c\ˆancer de mama aliada a detec\c{c}\~ao precoce contribuiu para que as taxas de mortalidade apresentassem redu\c{c}\~ao em alguns pa\'ıses nos \'ultimos anos [1]. No Brasil, as taxas de mortalidade continuam altas e a doen\c{c}a constitui a primeira causa de morte por c\ˆancer entre as mulheres [2, 3]. As taxas de mortalidade por c\ˆancer de mama no Brasil continuam elevadas devido a diagn\'osticos em est\'agios avan\c{c}ados da doen\c{c}a [2]. Dados do Minist\'erio da Sa\'ude indicam que cerca de 60 por cento dos tumores de mama s˜ao diagnosticados nos est\'agios III ou IV [2].  O diagn\'ostico tardio implica a redu\c{c}\~ao da sobrevida pela doen\c{c}a e um tratamento mais agressivo, aumentando o risco de complica\c{c}\~oes e rea\c{c}\~oes adversas. As rea\c{c}\~oes adversas s\~ao um importante componente que afeta a qualidade de vida dos pacientes, mas que podem por vezes ser evitadas ou atenuadas por meio de condutas terap\ˆeuticas individualizadas. Para isso, \'e importante o
conhecimento dos fatores relacionados ao desenvolvimento das rea\c{c}\~oes e com a ajuda de modelos preditivos j\'a adequar o tratamento quimioter\'apico tendo em vista as prov\'aveis rea\c{c}\~oes.


\textbf{67. OBJETIVO}

O objetivo geral \'e, então, propor um modelo computacional capaz de classificar e/ou quantificar probabil\'isticamente as rea\c{c}\~oes de novas portadoras em tratamento quimioter\'apico, tendo em vista as evid\^encias anteriormente usadas para ajuste do modelo. %missing values and infer evidence?

\vspace{0.4cm}
\textbf{7. METODOLOGIA}
%simple to complex, evaluating cover of all cases and F-score etc


\vspace{0.4cm}
\textbf{8. MATERIAIS}
%free-books, books, R software/packages and INCA dataset 
Relacionar os materiais que est�o previstos no projeto (\textit{computadores, instrumentos, equipamentos, dados, software: explicitar se h� licen�a})

\vspace{0.4cm}
\textbf{9. CRONOGRAMA}
%logistic regression, knn, svm, tree, bayesian
Apresentada graficamente conforme a Figura \ref{Fig:Cronograma}.

Fase 1: 

Fase 2: 

Fase 3: 

Fase 4: 

Fase 5: 

Fase 6: 

\begin{figure}
\begin{center}
\parbox[h]{14cm}
  {
  \begin{center}
  \includegraphics[scale=1.0]{Cronograma.eps}
  \caption[\small{(\textit{Aten��o, evitar projetos com menos de 5 meses})}]{\label{Fig:Cronograma} \footnotesize{(\textit{Aten��o, evitar projetos com menos de 5 meses})}}
  \end{center}
  }
\end{center}
\end{figure} 

\begin{thebibliography}{}

\bibitem{StMichael05}"Contrast and Contraindinctions" - Hospital St. Michael, Toronto, Canad�. http://www.stmichaelshospital.com/content/programs/medical imaging/, 2005, (Acesso em 12 Junho 2005).

\bibitem{Binmore92}BINMORE, Ken. "Fun and Games". Lexington, D. C. Heath, 1992. p.51.

\bibitem{Lins98}CARVAHO, Roberto Lins de, OLIVEIRA, Cl�udia Maria, "Modelos de Computa��o e Sistemas Formais". Rio de Janeiro, Universidade Federal do Rio de Janeiro, 1998.

%@phdthesis{paula2015metodos,
%  title={M{\'E}TODOS LONGITUDINAIS APLICADOS AO ESTUDO DA MULTICAUSALIDADE ASSOCIADA A REAC OES ADVERSAS AO TRATAMENTO QUIMIOTERAPICO PARA O CˆANCER DE MAMA},
%  author={Paula, Daniela Polessa},
%  year={2015},
%  school={Universidade Federal do Rio de Janeiro}
%}
%@book{james2013introduction,
%  title={An introduction to statistical learning},
%  author={James, Gareth and Witten, Daniela and Hastie, Trevor and Tibshirani, Robert},
%  volume={112},
%  year={2013},
%  publisher={Springer}
%}
%
%@book{pfeffer2016practical,
%  title={Practical Probabilistic Programming},
%  author={Pfeffer, Avi},
%  year={2016},
%  publisher={Manning Publications Co.}
%}
\end{thebibliography}

      \vspace{2cm}
      \noindent
Rio de Janeiro, 21 de julho de 2008

      \vspace{0.5cm}
      \begin{flushright}
         \parbox{10cm}{
            \hrulefill

            \vspace{-.375cm}
            \centering{Matheus Hoffmann Fernandes Santos}

            \vspace{0.9cm}
            \hrulefill

            \vspace{-.375cm}
            \centering{Alexandre Visintainer Pino}
 
            \vspace{0.9cm}
         }
      \end{flushright}
      \vfill
      
\end{document}
